\documentclass[11pt]{article}

\usepackage[a4paper,margin=2cm]{geometry}
\usepackage[dutch]{babel}
\usepackage{parskip}
\usepackage{amssymb}
\usepackage{amsthm}
\usepackage{amsmath}

\begin{document}

\begin{center}
\LARGE{Rechthoekige Problemen} \\
\end{center}

\section{Inleiding}


\section{Het probleem}
Het probleem luid als volgt:
\\
Beschouw een rooster van $m \times n$ hokjes. Dit rooster moet dan op zo een manier worden ingekleurd zodat voor ieder subvierkant het aantal gekleurde hokjes en het aantal lege hokjes beide minimaal voldoen aan de formule:
\[
Hokjes \geq \frac{x(x-1)}{2}.
\]
Hierin is $x$ de grote van een subvierkant.
\\
Verder moet voor het hele rooster gelden dat het hele gekleurde gebied een oppervlak vormt. Dus ieder gekleurd hokje zit vast aan een ander gekleurd hokje (Let op: niet diagonaal dit geeft namelijk een hele makkelijke oplossing). Dit zelfde geldt voor de lege hokjes.

\section{Proposities}
\subsection{2 bij 2 subvierkant}
Er kan met Grafentheorie bewezen worden dat het volgende niet kan. In de tabel betekent $1$ ingevuld en $0$ leeg. Dit geldt voor alle $2 \times 2$ subvierkanten:
\[
\begin{array}{|r|r|}
\hline
1 & 0 \\
\hline
0 & 1 \\
\hline
\end{array}
\]

\subsection{Gewoon handig}
Dit is eigenlijk geen propositie maar meer om snel $3 \times 3$ subvierkanten te controleren. Als u een "U" ziet is een $3 \times 3$ fout. Dus:
\[
\begin{array}{|r|r|r|}
\hline
1 & 1 & 1 \\
\hline
0 & 0 & 1 \\
\hline
1 & 1 & 1 \\
\hline
\end{array}
\]
Deze klopt niet omdat er niet aan de voorwaarde voor het aantal wordt voldaan (te weinig leeg: moet minstens 3). Deze is makkelijk te herkennen in een vermoede oplossing. Dit zelfde geldt als er een "H" wordt gezien. Dus:
\[
\begin{array}{|r|r|r|}
\hline
1 & 1 & 1 \\
\hline
0 & 1 & 0 \\
\hline
1 & 1 & 1 \\
\hline
\end{array}
\]
Deze twee gevallen betekenen natuurlijk niet dat dit de enige twee foute mogelijkheden zijn in een $3 \times 3$ subvierkant.



\end{document}